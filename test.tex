\documentclass[a4paper,11pt]{article}

\usepackage[latin1]{inputenc}
\usepackage{graphicx}
\usepackage{color}
\usepackage{soul}

%\usepackage[spanish]{babel}

\parskip=2mm



\begin{document}

\centerline{\Huge\sffamily\textbf{Star Trek}}

\medskip

 \centerline{\LARGE\textcolor{blue}{\textbf{Curiosidades sobre la propulsión Warp}}}

\bigskip

El empuje warp (empuje por curvatura; también conocido como ``impulso de deformación'' 
o ``de distorsión'') es una forma teórica de propulsión superlumínica. Este empuje permite
 propulsar una nave espacial a una velocidad equivalente a varios múltiplos de la 
velocidad de la luz, mientras se evitan los problemas asociados con la dilatación 
relativista del tiempo. Este tipo de propulsión se basa en curvar o distorsionar 
el Espacio-tiempo, de tal manera que permita a la nave ``acercarse'' al punto de 
destino. \ul{El empuje por curvatura no permite, ni es capaz de generar, un viaje 
instantáneo entre dos puntos a una velocidad infinita, tal y como ha sido sugerido 
en algunas obras de ciencia ficción, en las que se emplean tecnologías imaginarias 
como el ``hipermotor'' o ``motores de salto''.}

\bigskip

\noindent \ul{\large\textit{La invención del motor Warp}}
\medskip

En la historia de \textcolor{red}{\textbf{Star Trek}} se reconoce 
que el motor de curvatura fue inventado, en la Tierra, por Zefram 
Cochrane. La película \textbf{Star Trek: Primer Contacto} muestra como, 
en el año 2063, Cochrane realiza el primer viaje de curvatura de la especie humana, 
usando un antiguo misil nuclear intercontinental, modificado para viajar en el espacio 
y, una vez ahí, generar una burbuja Warp. Cochrane, para crear la burbuja Warp alrededor 
de la nave --y distorsionar el Espacio-tiempo para su desplazamiento-- precisó de una 
inmensa cantidad de energía (que obtuvo gracias a la reacción entre matería-antimateria). 
Este primer viaje supuso un hito, \hl{permitió alcanzar un factor de curvatura de 1,0} y 
condujo directamente al primer contacto con una raza extraterreste: \textsc{los vulcanos}.

\bigskip

\noindent \ul{\large\textit{Velocidad de curvatura. \textsc{Factor de curvatura}}}
\medskip

La unidad empleada con la velocidad de curvatura es el factor de curvatura 
ó ``warp factor''. La equivalencia entre factores de curvatura obtenidos por 
los reactores warp y velocidades medidas en múltiplos de la velocidad de 
la luz es en cierto modo ambigua.

Según la guía para escritores de episodios de Star Trek de la Serie Original, 
los factores warp se obtienen mediante la aplicación de la siguiente fórmula cúbica:
\begin{equation}
s(w) = w^3c
\end{equation}
donde \textit{w} es el factor warp, \textit{s(w)} es la velocidad medida 
en el espacio normal y 
\textit{c} es la velocidad de la luz. Según esta fórmula, `warp 1' 
es equivalente a la velocidad de la luz, `warp 2' equivale a 8 veces 
la velocidad de la luz, `warp 3' equivale a 27 veces la velocidad de la luz, etc.

\medskip

\hspace*{2mm} \ul{\textbf{Factor Warp, velocidad y tiempo de viaje a alpha centauri}} \\[3mm]
\hspace*{1mm} \textcolor{blue}{Factor Curvatura} \hspace{10mm} \textcolor{blue}{Velocidad (multiplos de c)} \hspace{10mm} 
\textcolor{blue}{Dias de viaje} \\[2mm]
\hspace*{15mm} 5 \hspace{45mm} 125 \hspace{35mm} 12,5 \\[1mm]
\hspace*{15mm} 8 \hspace{45mm} 512 \hspace{35mm} 4,1 \\[1mm]
\hspace*{15mm} 9.5 \hspace{42mm} 857 \hspace{35mm} 2,5 

\end{document}
